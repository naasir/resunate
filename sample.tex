%%% Local Variables:
%%% mode: latex
%%% TeX-master: t
%%% End:

%-----------------------------------------------------------
% content specific helper methods
%-----------------------------------------------------------

%-----------------------------------------------------------
% summary : helper command for specifying SLOC (source lines of code) counts
% usage   : \sloc{number of lines of code}
%-----------------------------------------------------------
\newcommand{\sloc}[1]{%
  {\footnotesize(#1 \href{http://en.wikipedia.org/wiki/Source_lines_of_code}{SLOC})}
}

%-----------------------------------------------------------
% content
%-----------------------------------------------------------
\documentclass{resunate}
\usepackage{pgfplots}

\begin{document}
\begin{resume}

  \begin{abstract}

    \name{Bruce Wayne}
    \phone{\href{tel:888-123-4567}{888-123-4567}}
    \email{bwayne@jokermail.com}
    \website{\href{http://bigbadbruce.jk}{bigbadbruce.jk}}

    \blurb{I'm a billionaire, playboy philanthropist with little to no real-world experience.
      I enjoy fast cars, cool technology, martial arts and the company of bats. My favorite color is black. }

    \maketitle

    \vspace{2.1in}
    \textit{In my spare time, I like to...}\\[5pt]
    \begin{tikzpicture}
      \pgfplotsset{
        x axis style/.style={
          xtick style=#1,
          x axis line style=#1,
        },
        y axis style/.style={
          ytick style=#1,
          y axis line style=#1,
        }
      }
      \begin{axis}[%
          xbar,
          bar width=5,
          height=1.5in,
          width=\textwidth,
          xmin=0,
          xtick=\empty,
          x axis style=white,
          xticklabels={a little, a lot},
          xtick={0,1},
          xticklabel style=gray,
          yticklabels={Laugh, Cook, Drive, Fight Crime},
          ytick={0,1,2,3},
          y tick label style={font=\small,text width=0.5in,align=right},
          y axis style=white,
          axis x line*=none,
          axis y line*=left,
        ]
        \addplot[draw=highlightcolor,fill=highlightcolor] coordinates {(1.0,3)(0.8,2)(0.5,1)(0.2,0)};
      \end{axis}
    \end{tikzpicture}

  \end{abstract}
  &
  \begin{body}

    \begin{section}{History}

      \workentry{Sept. 2008}
                {Present}
                {Software Developer}
                {WayneTech Systems}
                {Gotham City, NY}
                {Developed a range of products in the hi-tech and defense industry from the ground up, with a focus on web applications and services. Selected projects include:}

       \entry{WayneTech Secret API\textregistered}
             {C\# \sloc{25K} \wrench}
             {A public API for WayneTech's core super project.
               \begin{itemize}
                 \item Implemented as a RESTful JSON web service
                 \item Acted as the technical lead and was the top contributor
                 \item Built key features such as ad-hoc querying support, data-paging support and a lazy-loading client library
                 \item Built a BDD-styled integration test suite using \href{http://coffeescript.org}{Coffeescript} and \href{http://nodejs.org}{Node.js} as a means to validate and document the API
             \end{itemize}}

       \entry{WayneTech Secret Project \#1\textregistered}
             {C\# \sloc{2.5K}, Js \sloc{2.5K} \wrench}
             {A web application for managing communication with top secret satellites.
               \begin{itemize}
                 \item Implemented as a single-page, HTML5 web application using \href{http://backbonejs.org/}{Backbone.js}, \href{http://jquery.com/}{jQuery} and \href{http://lesscss.org/}{LESS}
                 \item Acted as the technical lead and was the top contributor
                 \item Developed the overall design of the UI
             \end{itemize}}

       \entry{WayneTech Secret Project \#2\textregistered}
             {C\# \sloc{21K}, Js \sloc{1.5K} \wrench}
             {A web application allowing people to make anonymous crime tips online.
               \begin{itemize}
                 \item Implemented as a SOAP-based web service and a multi-page web application using ASP.NET, \href{http://jquery.com/}{jQuery} and \href{http://lesscss.org/}{LESS}
                 \item Acted as the technical lead
                 \item Developed the overall design of the UI
             \end{itemize}}

    \end{section}

    \begin{section}{Artifacts}

      \vspace{5pt} % HACK: Eventually fix the spacing

      \entry{H.A.R.D.A.C (2005)}
            {Python \wrench}
            {Command-line based, natural language AI Agent (similar to the famous \href{http://en.wikipedia.org/wiki/ELIZA}{ELIZA}) designed to evaluate “Art”. Developed concept, personality, and computer vision aspects. Implemented \href{http://en.wikipedia.org/wiki/Sobel_operator}{Sobel} image edge detection algorithm and other metrics to assign value to user-provided image.}

      \entry{Transcend (2005)}
            {Js, SVG \wrench}
            {Web installation focused on translating physical gestures onto the Web. Developed mouse-gesture interface embedded directly into browser. Interpolated user-drawn paths into smooth Bezier curves and mapped these gestures as triggers for output messages.}

    \end{section}

    \begin{section}{Knowledge}

      \workentry{Oct. 2002}
                {Jun. 2008}
                {BFA in Digital Arts and Experimental Media \href{http://www.dxarts.washington.edu/}{(DXARTS)}}
                {University of Washington, Honors Program}
                {Seattle, WA}

    \end{section}

  \end{body}
\end{resume}
\end{document}
